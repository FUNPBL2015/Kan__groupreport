%\chapter{使用技術}
\section{印刷}
印刷するリーフレットの画像データをiPhoneの機能であるAirPrintを使って無線で飛ばし、リーフレットを印刷する。AirPrintを使用するためには画像データをNSDataにする必要があったためリーフレットの画像をpdfに変換し、そのデータをNSDataに変換することとした。両面印刷するためには2枚の画像データを1つのものとして扱う必要があったため、pdfにするプロセスを挟んだ。AirPrintに関するSwiftでの実装例があまりなく、Objective-Cでプログラムが書かれていることが多かったため実装は難しかった。そのため、Objective-Cで書かれたコードを1行ずつSwiftで書き直して実装した。
 印刷ボタンをタップした後は、AirPrintするための設定画面が表示される。そこではプリンターを選択、印刷枚数、白黒印刷にするのかカラー印刷にするのか選択することができる。ユーザが以上のことを選択し、印刷ボタンを選択すればAirPrintでデータがプリンターに送信され印刷が開始される。以下にそれぞれかくとなるコードをピックアップして記述した。
\begin{description}
\item[画像データをPDFにする処理]\mbox{} 
\begin{lstlisting}[basicstyle=\ttfamily\footnotesize, frame=single]
1.let path = arrayPaths[0] as NSString
2.let fullname : NSString = (filename as String) + ".pdf"
3.let pdfFilename = (path as String) + "/"+(fullname as String)
4.UIGraphicsBeginPDFContextToFile(pdfFilename, CGRectZero, nil)
5.UIGraphicsBeginPDFPageWithInfo(CGRectMake(0, 0, 1690, 1195), nil)
6.let point1 = CGPointMake(0, 0)
7.front.image?.drawAtPoint(point1)
8.UIGraphicsBeginPDFPageWithInfo(CGRectMake(0, 0, 1754, 1240), nil)
9.let point2 = CGPointMake(0, 0)
10.back.image?.drawAtPoint(point2)
11.UIGraphicsEndPDFContext()
12.myData = NSData(contentsOfFile: pdfFilename)!
 \end{lstlisting}
1. 保存するディレクトリのパスを宣言する\\
2. pdfファイル名を宣言する\\
3. パス名を含めたpdfファイル名の宣言する\\
4. pdfを作成するために詳細な設定を決めるために宣言する\\
5. 1枚目のpdfファイルはページの大きさを指定する\\
6. 1枚目のpdfファイルを書き込む座標の開始時点を指定する\\
7. 1枚目で指定した座標を設定に反映させる\\
8. 2枚目のpdfファイルはページの大きさを指定する\\
9. 2枚目のpdfファイルを書き込む座標の開始時点を指定する\\
10. 2枚目で指定した座標を設定に反映させる\\
11. pdfの処理を終える\\
12. pdfをNSDataに変換する\\
\item[印刷する処理]\mbox{} 
\begin{lstlisting}[basicstyle=\ttfamily\footnotesize, frame=single]
@IBAction func print(sender: AnyObject) {
       1. if(UIPrintInteractionController.canPrintData(myData)){
           2. let printController = UIPrintInteractionController.
            .sharedPrintController()
            3.printInfo.outputType = UIPrintInfoOutputType.General
            4.printInfo.duplex = UIPrintInfoDuplex.LongEdge
            5.printInfo.orientation = UIPrintInfoOrientation.Landscape
            6.printController.showsPageRange = true
            7.printController.printingItem = myData
            8.printController.presentAnimated(true, 
            completionHandler: nil)
       }
}
 \end{lstlisting}
 1. 取得したデータがAirPrintできるデータであれば実行する\\
 2. AirPrintの設定ができる画面を宣言する \\
 3. 印刷をカラーか白黒か選べるようにする \\
 4. 両面印刷の長辺とじに設定する \\
 5. 印刷する向きを横向きにする \\
 6. 印刷する向きを反映する \\
 7. プリントするデータを設定する \\
 8. AirPrintの設定ができる画面に遷移する \\
\end{description}

\bunseki{池田俊輝}