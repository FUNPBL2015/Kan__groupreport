%\chapter{学び}
\section{グループ間の情報共有}
前期からプロジェクトの時間の最後の20分は各グループで進捗報告を行うようにしていた。報告の内容は、その日に行った活動とその活動がどういった進捗を生んだのかを報告し合うものであった。前期は、どのグループも基本は要件定義の段階で止まっていたため、質問があるかと聞かれてもいつまでに誰がどのようなことを行うのか程度の受け答えしか行っていなかった。まずここに第一の反省点が見受けられた。今どのようなことに詰まってて、どう解決していいのか分からなくて困っているかを相談することが無かったということだ。このような相談をグループ間で行っておけばもしかしたら、何か別の解決策が生まれて早期段階で問題解決に近づけたのかもしれなかった。
\par 後期、第3サイクルの時点では我々観光グループと医療グループは実装に取り掛かり始めていた。しかし、進捗報告の時間では出来たものを見せ合うことはしておらず、依然としてただプロジェクトの活動中に行ったことを報告し合うだけでいた。ここで第二の反省点が見られた。担当教員に指摘されるまでアプリを公にして見せ合うことがなかったため、同じ技術をそれぞれのグループが利用していて、片方が習得しているのにもう片方が詰まっていて、それに気づくことに遅れてしまったのだ。具体的には両グループ共にカメラを用いた技術を利用していた。こちらがカメラの機能の使用画面の言語が英語になってしまい、その問題の解決策が見出せないままでいた。休憩時間にメンバーが医療グループにアプリを見せてもらいに行ったとき、医療グループでも我々と同じカメラ機能を実装しており、さらにこちらの抱えていた問題を解決済みであった。それにより、お互いに解決をすることが出来た、ということがあった。
\par 第4サイクルに入ってからは、前サイクルで担当教員に「アプリの画面が出来ているなら、出来ているところまででもプロジェクターとかで映してデモお見せたほうが良い。」とアドバイスを頂いたので、それを実行した。これにより、TAや教員だけでなく、学生間でのアドバイスを行えるようになった。それは技術面だけに限らず、ボタン・画像の配置やアプリ内のデザインの配色等のUI面でもお互いに見直せる機会が増えた。
\par プロジェクトの活動は終わってしまったが、情報共有をしておくことは今後も活かせる学びであると言えるだろう。研究室ではメンバーや教員と、就職先では本物のプロジェクトとして活動をしていくことになる。その時、今回の反省点である相談をすることと報告・連絡をすることが重要になってくるだろう。
\bunseki{山川拓也}