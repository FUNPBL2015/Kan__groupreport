\subsection{写真撮影画面}
本アプリにはiPhoneに内蔵しているカメラを使用して現地の写真を撮影する機能がある。カードリストをユーザが見て、観光地のおすすめスポットに観光しに行った時に現地の様子を写真に残してもらうために実装した。\\
 まず最初に”AVFoundation Framework”というswiftのFrameworkを使用して実装しようとした。このFrameworkは開発者がカメラの画素数やピントの調整のやり方などを全てマニュアルで制御しなくてはならなかった。とくにピントの調整においては、撮影の対象物とiPhoneのカメラの距離から自分で計算式を考えてピントが合うようにしなければならずうまく実装することができなかった。\\
 次に”AVFoundation Framework”を使わず”UIImagePickerController”というクラスを用いて実装した。このクラスを用いるとピントの調整や明るさの調整はアプリ側がオートで処理をしてくれるため簡単にカメラで写真をとる機能を実装することができた。そして、撮影した写真はUIImageとして保存されデータベースに書き込むことができる型ではなかったため、UIImageデータをNSDataに変換してアプリ内のデータベースに保存することにした。これにより、他のクラスで撮った画像を使用する場合もアクセスが可能になった。\\
 最終的にこのアプリの機能として、ユーザはカードリストからカメラボタンをタップするとカメラが起動する。そこで写真を撮影すると、その写真を使用するのかキャンセルするのか選択する画面が表示される。写真を選択するボタンをタップすると、カードリスト画面まで遷移してカードリストの写真が撮った写真に変更されるようになる。
\bunseki{池田俊輝}