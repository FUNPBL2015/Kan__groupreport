% プロジェクト学習中間報告書書式テンプレート ver.1.0 (iso-2022-jp)

% 両面印刷する場合は `openany' を削除する
\documentclass[openany,11pt,papersize]{jsbook}



% 報告書提出用スタイルファイル
\usepackage[final]{funpro}%最終報告書
%\usepackage[middle]{funpro}%中間報告書

% 画像ファイル (EPS, EPDF, PNG) を読み込むために
\usepackage[dvipdfmx]{graphicx,color}

% ここから -->
\usepackage{calc,ifthen}
\newcounter{hoge}
\newcommand{\fake}[1]{\whiledo{\thehoge<70}{#1\stepcounter{hoge}}%
  \setcounter{hoge}{0}}
% <-- ここまで 削除してもよい

% 年度の指定
\thisYear{2015}

% プロジェクト名
\jProjectName{フィールドから創る地域・社会のためのスウィフトなアプリ開発}

% [簡易版のプロジェクト名]{正式なプロジェクト名}
% 欧文のプロジェクト名が極端に長い(2行を超える)場合は、短い記述を
% 任意引数として渡す。
%\eProjectName[Making Delicious curry]{How to make delicious curry of Hakodate}
\eProjectName{``Swift'' Application Development Based on Field Research}


% <プロジェクト番号>-<グループ名>
\ProjectNumber{3-A}

% グループ名
\jGroupName{観光系グループ}
\eGroupName{Tourism Group}

% プロジェクトリーダ
\ProjectLeader{1013220}{新保遥平}{Yohei~Shinpo}

% グループリーダ
\GroupLeader  {1013068}{岩見建汰}{Kenta~Iwami}

% メンバー数
\SumOfMembers{5}
% グループメンバ
\GroupMember  {1}{1013001}{池田俊輝}{Toshiki~Ikeda}
\GroupMember  {2}{1013068}{岩見建汰}{Kenta~Iwami}
\GroupMember  {3}{1013167}{山川拓也}{Takuya~Yamakawa}
\GroupMember  {4}{1013224}{細川椋太}{Ryota~Hosokawa}
\GroupMember  {5}{1013228}{横山翔栄}{Shoei~Yokoyama}

% 指導教員
\jadvisor{伊藤恵,奥野拓,原田泰,木塚あゆみ,南部美砂子}
% 複数人数いる場合はカンマ(,)で区切る。カンマの前後に空白は入れない。
\eadvisor{Kei~Itou,Taku~Okuno,Yasushi~Harada,Ayumi~Kizuka,Misako~Nambu}

% 論文提出日
\jdate{2015年7月29日}
\edate{July~29, 2015}

\begin{document}
%
% 表紙
\maketitle

%前付け
\frontmatter

\input abstract.tex

\tableofcontents% 目次

\mainmatter% 本文のはじまり

\chapter{背景}
\input background.tex

\chapter{到達目標}
\input goal.tex

\chapter{プロジェクトの進め方}
\input project_procedure.tex

\chapter{開発準備}
\input develop_preparation.tex

\chapter{開発プロセス}
\input develop_process1.tex
\input develop_process2.tex
\input develop_process3.tex
\input develop_process4.tex

\chapter{キーコ紀行について}
\section{キーコ紀行の概要}
\input kiko_abstract.tex
\section{「観光する」機能}
\input kiko_cardlist.tex
\input kiko_map.tex
\input kiko_takephoto.tex
\input kiko_textedit.tex
\section{「振り返る」機能}
\input kiko_album.tex
\section{「印刷する」機能}
\input kiko_print.tex
\section{リーフレットについて}
\input kiko_leaflet.tex

\chapter{今後の課題と展望}
\input release.tex
\input connection.tex

\chapter{学び}
\input learned_poster.tex
\input learned_infoshare.tex
\input learned_planmanagement.tex
\input learned_version_management.tex

\chapter{まとめ}
\input summary.tex
% 参考文献
\begin{thebibliography}{9}
\bibitem{NN2013}
\newblock 西村直人,永瀬美穂,吉羽龍太郎
\newblock SCRUM BOOT CAMP THE BOOK.
\newblock 株式会社翔泳社, 2013.

\bibitem{YM2014}
\newblock 森巧尚.
\newblock Swift ではじめるiPhone アプリ開発の教科書.
\newblock 株式会社マイナビ, 2014.

\bibitem{HO2014}
\newblock 大塚弘記.
\newblock WEB+DB PRESS plusシリーズ GitHub実践入門 -Pull Requestによる開発の変革.
\newblock 株式会社技術評論社, 2014.

\bibitem{HH1980}
佐賀市南部周遊バスで行く SAGAARIAKEガタバトルコース. 佐賀市. \par
http://www.saga-city.jp/gatabattle/tour.html. (2015/7/22 アクセス)

\bibitem{YI2003}
いさりび鉄道に事業許可 国交省 来春開業へ準備本格化. どうしんウェブ/電子版, 2015.\par
http://dd.hokkaido-np.co.jp/news/politics/politics/1-0151087.html. (2015/7/22 アクセス)

\bibitem{KI1989}
北海道情報誌 HO[ほ] バックナンバー. 株式会社 ぶらんとマガジン社, 2015.\par 
http://www.toho-ho.jp/backnumber. (2015/7/22 アクセス)

\bibitem{TK1995}
デザインの「まとめる」を、ちょびっとかじる。.  	アトリエ・カプリス. \par
http://dzukai.com/egokoro/ws\_matome\_ex.html. (2015/7/22 アクセス)

\bibitem{TK1993}
Japan - The Strange Country (Japanese ver.). vimeo.com, 2010.\par
https://vimeo.com/9873910. (2015/7/22 アクセス)

\bibitem{KK2004}
統計データを「心に響く動画」に変える”ビデオインフォグラフィック”とは?. movieTIMES, 2014.\par
http://www.movie-times.tv/purpose/branding\_idea/4366/. (2015/7/22 アクセス)

\bibitem{NK1994}
クールなインフォグラフィック動画15選. HepHep!, 2013.\par
http://hep.eiz.jp/201306/infographics/. (2015/7/22 アクセス)

\bibitem{NM1998}
木古内町で記念事業続々道新150604.pdf. 北海道新聞, 2015.\par
https://www.dropbox.com/s/juxgayyd1y7h3eu/%E6%9C%A8%E5%8F%A4%E5%86%85%E7%94%BA%E3%81%A7%E8%A8%98%E5%BF%B5%E4%BA%8B%E6%A5%AD%E7%B6%9A%E3%80%85_%E9%81%93%E6%96%B0150604.pdf?dl=0
. 
(2015/7/22 アクセス)

\bibitem{SN1991}
日曜トーク道新150531.pdf. 北海道新聞, 2015.\par
https://www.dropbox.com/s/ic7ew6fvxospnp9/%E6%97%A5%E6%9B%9C%E3%83%88%E3%83%BC%E3%82%AF_%E9%81%93%E6%96%B0150531.pdf?dl=0
. (2015/7/22 アクセス)

\bibitem{NO2000}
5分で分かる、「スクラム」の基本まとめ. atmarkIT, 2012.\par
http://www.atmarkit.co.jp/ait/articles/1208/07/news128.html. (2015/7/22 アクセス)

\bibitem{CQ1991}
町長がメールで木古内PR. 函館新聞, 2015.\par
http://www.hokkaido-nl.jp/detail.cgi?id=26392. (2015/7/22 アクセス)

\bibitem{IS1991}
Swiftで写真からGPS情報を取得. psychosistem.jp, 2014.\par
http://psychosistem.jp/blog/?p=934. (2015/7/22 アクセス)

\bibitem{}
Googleマップで緯度経度を求める. numazu-ct.ac.jp.\par
http://user.numazu-ct.ac.jp/-tsato/webmap/sphere/coordinates/advanced.html. (2015/7/22 アクセス)

\bibitem{}
SwiftでAppDelegateを使った画面間のデータ受け渡し. Qiita.com, 2014.\par 
http://qiita.com/xa\_un/items/814a5cd4472674640f58. (2015/7/22 アクセス)

\bibitem{}
アノテーションに画像を追加. SwiftDocs, 2015. \par
https://sites.google.com/a/gclue.jp/swift-docs/ni-yinki100-ios/7-mapkit/anoteshonni-hua-xiangwo-zhui-jia. (2015/7/22 アクセス)

\bibitem{}
Swiftコードで画面遷移させたいときの方法. cheekpouch.com, 2014.\par
http://blog.cheekpouch.com/352/. (2015/7/22 アクセス)

\bibitem{}
実践! iPhoneアプリ開発. マイナビニュース, 2009.\par
http://news.mynavi.jp/column/iphone/017/. (2015/7/22 アクセス)

\bibitem{}
木古内町観光協会. 木古内町.\par
http://kikonai-kankou.net/. (2015/7/22 アクセス)
\end{thebibliography}

\end{document}