%\chapter{学び}
\section{計画管理の必要性}
プロジェクト学習を通して計画することとそれを継続的に管理することの重要さを学んだ。計画の管理をうまく行えていなかったことによる最も大きな失敗は、第4サイクルでキーコ紀行の実装を行ったときにあった。第4サイクルの開発開始時にタスクの洗い出しを行い、そのタスクの管理のためにWBS(ワークブレークダウンストラクチャ)を作り運用のルールなども決めたが、それを継続的に運用するには至らなかった。そのため、どの機能を実装するのか、その優先度はどの程度か、それを誰がやるのか、といった計画の管理が上手くいかず、それぞれのメンバーに認識のずれが生じてしまったことがあった。また、メンバーそれぞれが現在持っているタスクの把握もできていなかった。その認識のずれの解消やタスクの把握をするために本来必要でなかったはずのコミュニケーションに時間を割いてしまった経験から、継続的に計画管理を行うことの必要性に気付いた。理想的には、WBSやかんばんなどの形式に則ったタスク管理を、チームのメンバー全員が行うべきだと考えるが、継続的に運用するためには、その重要性を周知し、運用を習慣化させるとともに、より手間をかけずに管理できるような工夫が必要である。しかし、具体的に運用方法をどうすべきかという点に関しては未だ不明瞭なままであるため、その枠組みを考えることが今後の課題である。
\bunseki{細川椋太}