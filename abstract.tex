\begin{jabstract}
\quad 本プロジェクトでは、まずフィールドを実際に調査してそこから問題点を見つける。そこで見つかった問題点を解決するためにiOSアプリケーション(以下、アプリとする)を開発して、それにより地域・社会に貢献することを目標として活動を行っている。開発手法はアジャイル開発を用いる。素早くアプリを開発し、それに対するレビューを受けてさらに質の高いアプリを開発していく。\\
\quad 我々観光系グループは2016年春の北海道新幹線開業に伴い、観光産業に力を入れ始めている木古内町をフィールドに設定した。木古内駅は北海道新幹線の停車駅の一つとなる。新幹線開業によって、TVやSNS等で取り上げられており、一定の知名度上昇は見込まれる。しかしながら、観光客を呼び込むには魅力を伝える手段が必要であると考えられる。そこで、フィールドから木古内町の魅力や抱える問題点を探しながら、木古内観光をより魅力的に、より便利にするアプリの開発を行う。前期の活動として、アプリを開発していく上で必要な技術の勉強会を行ってメンバー全員が基本的な技術を得た。また、同時期に実際に木古内町に現地調査へ行き、どのような問題点があるのか、どのような魅力があるのかを調査した。その結果我々がどのように感じたか、どのようなアプリが必要なのかを話し合った。我々が話し合って固めたアプリ案をティーチングアシスタント(以下、TAとする)や先生、企業の方々に見てもらってアドバイスをもらい、7月にアプリのver.1.0を開発した。そこからさらにレビューを受けてver.2.0を開発してプロジェクト学習の中間発表会で発表した。\\
\quad これまでの活動を通して生じた課題は、まだアプリの機能がありふれたものであり、木古内であるオリジナリティがないことだ。また、フィールドから得たことを活かせていないのが現状の重大な課題だ。一方、タスク管理をしておくことで作業をスムーズに把握できるようになったことや、これまでのアプリ案の提案やポスター作成においてレビューをもらうことでより質の高いものを制作できるようになることなどの学びを得た。\\
\quad 今後の活動としては、アプリのver.3.0以降の開発を行っていく。また、木古内町の関係者に実際にアプリを見てもらってレビューをもらい、新たなアプリの機能案やこれまで作ってきた機能の拡張案を見つけていく予定だ。

%\fake{ここに日本語の概要を書きます。}
% 和文キーワード
\begin{jkeyword}
木古内町,観光,アジャイル開発, iOSアプリ,現地調査,レビュー
\end{jkeyword}
\bunseki{山川拓也}
\end{jabstract}

%英語の概要
\begin{eabstract} 

\ In this project, at first, we really investigate a field and find problems from field survey. We develop iOS application(app) to solve problems found from field survey. We are thereby active with the goal of contributing to an area. We use Agile development which is a development technique. We do swift app development, and develop higher quality app by receiving the reviews for it. 

\ Our tourism group set Kikonai town as the field. The town began to lay emphasis on tourism industry due to opening Hokkaido Shinkansen in the spring of 2016. Kikonai station will become one of the stations of Hokkaido Shinkansen. TV or SNS take up about it by the Shinkansen opening of business, and the rise of fixed popularity is anticipated. However, as for us, it is thought that means to introduce charm into is necessary to call in tourists. Then, we develop app to do Kikonai tourism more usefully and attractively while looking for charm of the town and problems to have from there. As first-term activity, the members performed the study session of a necessary technique in developing app and learned basic techniques. In the same period, we also surveyed it what kind of problems and what kind of charms it has by going to the town in fact. From the survey, we discussed what kind of app was necessary and how did we feel it. We received many advices about our app's ideas from Teaching Assistant(TA), teachers, and the company people. Then, in July, we developed ver.1.0 of app. We received reviews more from there and developed ver.2.0 of it and announced it at a middle presentation of the project learning. 

\ The problems that occurred through past activity are that app's functions was still occurred and there is not originality for Kikonai. Also, we could not make use of experience from the field. However, we obtained learning, for example, grasp of the work became more smooth by doing task management and we became able to produce higher quality things by getting reviews in suggestion of app's plan and poster making. 

\ As future activity, we develop after ver.3.0 of the app. Also, we will receive reviews about our app from people concerned to Kikonai. Then, we are going to find function plans and expansion plans of new app's functions.


% 英文キーワード
\begin{ekeyword}
Kikonai Town, Tourism, Agile development, iOS Application, Field Survey, Reviews
\end{ekeyword}
\bunseki{山川拓也}
\end{eabstract}