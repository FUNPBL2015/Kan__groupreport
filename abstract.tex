\begin{jabstract}
\quad 本プロジェクトでは、フィールドを実際に調査してそこから問題点を見つける。そこで見つかった問題点を解決するためにiOSアプリケーション(以降、アプリとする)を開発して、それにより地域・社会に貢献することを目標として活動を行っている。開発手法はアジャイル開発手法を用いる。素早くアプリを開発し、それに対するレビューを受けてより質の高いアプリを開発していく。\\
\quad 我々観光系グループは2016年春の北海道新幹線開業に伴い、観光産業に力を入れ始めている木古内町をフィールドに設定した。5月初旬に実際に木古内町に現地調査へ行き、どのような問題点があるのか、どのような魅力があるのかを調査した。フィールドワークから、観光情報がSNSやWeb等で分散して分かりにくいという問題点と、観光客はスマートフォンで撮影した写真で思い出を振り返ることが少ないという問題点があることに気づいた。そこから我々が話し合って固めたアプリ案をティーチングアシスタント(以下、TAとする)や先生、企業の方々にレビューをもらいながら、PDCAサイクルを用いて前期と後期で4度の開発を行った。前期の開発では、分散している観光情報の集約と、観光中に撮影した写真を自動的にアルバムにして思い出話を盛り上げるための機能「フォトストーリ―」を提案した。しかし、7月の中間報告会で、これはただ観光情報を表示しているだけでユーザが「行きたい」と思えないという課題と、写真を撮るモチベーションが高くならないという課題が見つかった。後期では、前期の課題を解決するために、木古内で「できること」に着目した情報の表示、そして自分の撮った写真からオリジナルのリーフレットを作成する機能を提案し、実装を行った。この時、開発アプリの名前を「キーコ紀行」とした。12月の最終報告会ではキーコ紀行のデモを行い、レビューを受けた。\\
\quad 最終報告会でのキーコ紀行の評価が高かったため、今後もメンバー間で開発を進めて来春にはリリースすることを展望としている。まずは木古内の関係者にアプリをレビューしてもらうことが必要である。さらに、まだアプリ内に残っているバグ修正を行っていく。\\

%\fake{ここに日本語の概要を書きます。}
% 和文キーワード
\begin{jkeyword}
木古内町,観光,アジャイル開発, iOSアプリ,現地調査,レビュー, PDCAサイクル
\end{jkeyword}
\bunseki{山川拓也}
\end{jabstract}

%英語の概要
\begin{eabstract} 

\ In this project, at first, we really investigate a field and find problems from field survey. We develop iOS application(app) to solve problems found from field survey. We are thereby active with the goal of contributing to an area. We use Agile development which is a development technique. We do swift app development, and develop higher quality app by receiving the reviews for it. 

\ Our tourism group set Kikonai town as the field. The town began to lay emphasis on tourism industry due to opening Hokkaido Shinkansen in the spring of 2016. In the beginning of May, we surveyed Kikonai what kind of problems and charms it has by going to the town in fact. From this, we found that the sightseeing information disperses in SNS and web. Furthermore, tourists The tourists don't look back the photos which they took. Then, we developed an app four times totally with PDCA cycle in the first semester and the second semester while receiving many advices of our app's ideas to Teaching Assistant(TA), teachers, and the company people. In the first semester, we suggested two functions that intensiveness of dispersing information and making an album automatically from photos which users took during sightseeing in order to enliven recollections talk as ``photo story.'' However, in the midterm presentation of July, we found two problems that we only showed information about Kikonai and this app couldn't make user raise the motivation to take photos. In the second semester, we suggest two functions that sightseeing information paid its attention to ``being able to do it'' in Kikonai and making an original leaflet by using taken photos. Then we implemented them. In this time, we named this app ``Ki-ko Kiko.'' In the final presentation of December, we gave a demonstration of it and received reviews from many people.

\ We continue to develop it and intend to release this app by the next spring because its evaluation was high in the final presentation. At first it's necessary to have people concerned of Kikonai watch it. In addition, it's also necessary for us to revise bugs of it.



% 英文キーワード
\begin{ekeyword}
Kikonai Town, Tourism, Agile development, iOS Application, Field Survey, Reviews, PDCA Cycle
\end{ekeyword}
\bunseki{山川拓也}
\end{eabstract}