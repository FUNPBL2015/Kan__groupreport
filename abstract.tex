\begin{jabstract}
\quad 本プロジェクトでは、フィールドを実際に調査してそこから問題点を見つける。そこで見つかった問題点を解決するためにiOSアプリケーション(以降、アプリとする)を開発して、それにより地域・社会に貢献することを目標として活動を行っている。開発手法はアジャイル開発手法を用いる。素早くアプリを開発し、それに対するレビューを受けてより質の高いアプリを開発していく。\\
\quad 我々観光系グループは2016年春の北海道新幹線開業に伴い、観光産業に力を入れている木古内町をフィールドに設定した。5月初旬に実際に木古内町に現地調査へ行き、どのような問題点があるのか、どのような魅力があるのかを調査した。フィールドワークから、観光情報がSNSやWeb等に分散して分かりにくいという問題点があることに気づいた。また、観光客はスマートフォンで撮影した写真で思い出を振り返ることが少ないことに気づいた。そこから我々が話し合って固めたアプリ案をティーチングアシスタント(以降、TAとする)や担当教員、企業講師の方々にレビューを受けながら、PDCAサイクルを用いて前期と後期で4回の開発を行った。前期の開発では、分散している観光情報の集約と、観光中に撮影した写真を自動的にアルバムにして思い出話を盛り上げるための機能「フォトストーリ―」を提案した。しかし、7月の中間発表会で、これはただ観光情報を表示しているだけでユーザが「行きたい」と思えないという課題と、写真を撮るモチベーションが高くならないという課題が見つかった。後期では、前期の課題を解決するために、木古内で「できること」に着目した情報の表示、そして自分の撮った写真からオリジナルのリーフレットを作成する機能を提案し、実装を行った。この時、開発アプリの名前を「キーコ紀行」とした。12月の成果発表会ではキーコ紀行のデモを行い、レビューを受けた。\\
\quad 成果発表会でのキーコ紀行の評価が高かったため、今後もメンバー間で開発を進めて2016年2月にはリリースすることを展望としている。まずは木古内の町長や観光協会の人々にアプリをレビューしてもらうことが必要である。さらに、まだアプリ内に残っているバグ修正を行っていく。\\

%\fake{ここに日本語の概要を書きます。}
% 和文キーワード
\begin{jkeyword}
木古内町,観光,アジャイル開発, iOSアプリ,現地調査,レビュー, PDCAサイクル
\end{jkeyword}
\bunseki{山川拓也}
\end{jabstract}

%英語の概要
\begin{eabstract} 

\ In this project, at first, we investigate on the field and find problems from field survey. We develop iOS application (app) to solve the problems found from field survey. Then we have action with the goal of contributing to an area. We use Agile development process which is a software development technique. We do swift app development, and develop higher quality app by being reviewed for it. 

\ We tourism group set Kikonai town as the field. The town began to lay emphasis on tourism industry due to opening Hokkaido Shinkansen in the spring of 2016. In the beginning of May, we surveyed what kind of problems and charms Kikonai has by going there in fact. From this, we found that the sightseeing information disperses in SNS and Web. Furthermore, tourists don't look back the photos which they took. Then, we did development four times totally with PDCA cycle in the first semester and the second semester while being advised of our app's ideas to Teaching Assistant (TA), teachers, and the professional instructors from companies. In the first semester, we proposed two functions: intensiveness of dispersing information and ``photo story,'' making an album automatically from photos which users took during sightseeing in order to enliven recollections talk. However, in the midterm presentation of July, we found two problems that app only showed information about Kikonai and it couldn't make user raise the motivation to take photos. In the second semester, we proposed two functions:  sightseeing information focused on ``things to do in Kikonai'' and making an original leaflet by using taken photos. Then we implemented them. In this time, we named this app ``Ki-ko Kiko.'' In the final presentation in December, we gave a demonstration of it and have been reviewed from many people.

\ We continue to develop it and intend to release this app by February in 2016 because its evaluation was high opinion in the final presentation. At first it is necessary to have the mayor of Kikonai and tourism association review it. In addition, we will fix bugs in it.



% 英文キーワード
\begin{ekeyword}
Kikonai Town, Tourism, Agile Development Process, iOS Application, Field Survey, Reviews, PDCA Cycle
\end{ekeyword}
\bunseki{山川拓也}
\end{eabstract}