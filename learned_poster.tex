%\chapter{学び}
\section{ポスター}
本プロジェクトを通して、成果発表のために計3回のポスター製作を行った。これらのポスターについてもアプリと同様にレビューを受け、改善を図っていった。その中で、空白や図表の必要性などグラフィックデザインの要素について学ぶことができた。まず、空白とは何もない無のスペースではなく、複数の要素のグループ化や構造化などの役割を持つ要素の一つであることが分かった。これはゲシュタルト心理学におけるプレグナンツの法則のうち近接の要因と関係し、この法則については昨年度の認知心理学の講義で学習したものであるが、プロジェクトの中で行ったポスター製作によって実践的な学びを得ることができた。また、ポスターを図表を中心に組み立て、文章の分量を抑えることで、誘目性が高く魅力的なポスターになることが分かった。文章の分量が多いポスターは閲覧者に読む意欲を失わせる。これはプロジェクト成果を伝え、レビューをもらう機会を失うことにつながる。一方で、文章の分量を極力抑え図表を多く使用したポスターは、閲覧者に重い印象を与えず、気軽に読んでみようという意欲を起こさせる。とりわけポスターセッションにおいては図表を用いた口頭発表を行うため、冗長な文章は不要である。簡潔な文章と図によって、ポスターは読まれやすく、伝わりやすいものになることを学んだ。
\bunseki{横山翔栄}