%\chapter{到達目標に対する評価}
\section{機能の評価}
アプリケーションに実装した機能は、観光情報へのアクセス性を高め、撮影した写真を用いて観光の体験の振り返りを促すことを目標としていた。観光情報へのアクセス性を向上させるために実装した「観光する」機能は、発表会などで得られたレビューにより、既存のWebページやSNSと比較し情報を得やすいとの評価を得た。以下に、最終発表会で行ったアンケートの「このアプリの良いと感じた点は何ですか?」の項目から、観光情報へのアクセス性に関するコメントを抜粋した。
\begin{itemize}
\item 木古内で出来ることについて、グルメや観光スポットについて幅広く見れる
\item 観光地の内容がわかりやすく伝わること
\end{itemize}
これらのコメントから、観光情報へのアクセス性の向上についてはおおむね到達目標に達したものと評価できる。しかしながら、「今自分がいる場所から近い場所を探したい」「検索機能が欲しい」などの課題の指摘も受けた。今後、リリースへ向けた開発にあたっては、これらのレビューを踏まえて開発を行っていく。以下に、同アンケートの「このアプリの悪いと感じた点は何ですか?」の項目から観光情報へのアクセス性に関するコメントを抜粋した。
\begin{itemize}
\item 検索機能がないので、分野別に見たい時は使いにくい。
\item アクセスの仕方も見れると良かった
\end{itemize}
撮影した写真を用いて観光の体験の振り返りを促すために実装した「印刷する」機能は、自分で撮影した写真でリーフレットを作成できる点が「新しい」「思い出を形に残せる」との評価を得た。以下は同アンケートの「このアプリの良いと感じた点は何ですか?」の項目から、「印刷する」機能に関するコメントの抜粋である。
\begin{itemize}
\item データだけではなく物で思い出を残せる点
\item ありそうでないアプリ 着眼点は面白い
\item 自分好みのリーフレットを作れたり、「自分だけ」のが作れるのがいいと感じた。
\end{itemize}
以上から、機能に有用性があり、ユーザの興味関心を引くことができており、写真を用いた観光の振り返りを十分に促すという目標に到達したと評価できる。ただし、「自分でレイアウトを作りたい」「新規ユーザの獲得はどうするか」といった指摘もあり、これらの解決が今後の課題となっている。以下は同アンケートの「このアプリの悪いと感じた点は何ですか?」の項目から、「印刷する」機能に関するコメントの抜粋である。
\begin{itemize}
\item 思い出はできるが、新規の人が来ない。
\item 元の写真に戻すときのやり方がわからん
\item 自分なりのレイアウトが作れない
\end{itemize}

\section{活動内容の評価}

\bunseki{横山翔栄}