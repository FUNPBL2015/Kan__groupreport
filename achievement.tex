%\chapter{到達目標に対する評価}
\section{機能の評価}
アプリケーションに実装した機能は、観光情報へのアクセス性を高め、撮影した写真を用いて観光で得られる体験に付加価値を与えることを目標としていた。観光情報へのアクセス性を向上させるために実装した「観光する」機能は、発表会などで得られたレビューにより、既存のWebページやSNSと比較し情報を得やすいとの評価を得た。以下に、最終発表会で行ったアンケートの「このアプリの良いと感じた点は何ですか?」の項目から、観光情報へのアクセス性に関するコメントを抜粋した。
\begin{itemize}
\item 木古内で出来ることについて、グルメや観光スポットについて幅広く見れる
\item 観光地の内容がわかりやすく伝わること
\end{itemize}
これらのコメントから、観光情報へのアクセス性の向上についてはおおむね到達目標に達したものと評価できる。しかしながら、「今自分がいる場所から近い場所を探したい」「検索機能が欲しい」などの課題の指摘も受けた。今後、リリースへ向けた開発にあたっては、これらのレビューを踏まえて開発を行っていく。以下に、同アンケートの「このアプリの悪いと感じた点は何ですか?」の項目から観光情報へのアクセス性に関するコメントを抜粋した。
\begin{itemize}
\item 検索機能がないので、分野別に見たい時は使いにくい。
\item アクセスの仕方も見れると良かった
\end{itemize}


\section{活動内容の評価}

\bunseki{横山翔栄}