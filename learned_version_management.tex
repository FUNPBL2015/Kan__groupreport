%\chapter{学び}
\section{バージョン管理}
プロジェクト発足時にGitHub,Gitを用いたソースコードのバージョン管理についての勉強会に参加した。勉強会ではGitコマンドの使用方法を実践的に学び、第1サイクルではGitを使用してバージョン管理を行うことができた。また、Gitで使用するコマンドをマニュアルとして作成しチームで共有した。
\par 第2サイクルは中間発表まで1週間を切っていた、コーディング担当者である2人は日常的に会う頻度が高いメンバー同士だった、GitHub,Gitに対する知識が浅く第1サイクル時にマージした際にファイルが消えるなどのトラブルへの対処法が分からないままだったという要因が重なり、GoogleDriveやMacに搭載されているAirDropを使用してファイル共有を行いコーディングを進めていた。その際は入念にマージや分担について話しあったので大きなトラブルは発生しなかったが、時間を大幅に消費してしまった。その消費してしまった時間をコーディングや成果物の改善に関する議論に費やすことができたら中間成果発表時での成果物の質を高くすることができたと反省している。そして夏季休業終了後の第3サイクルへと突入したが、コーディングを行っていないためGitHub,Gitは使用していない。
\par 第4サイクルはGitHub,Gitを使用してコーディングを進めることができたが、GitHub,Gitを久々に使用するということもあり導入に戸惑ってしまった。原因としてGitHub,Gitに関する技術・知識の量がメンバー間で異なっていた事実を早めに解決すべきだったことがあげられる。GitHub,Gitを使用してコーディングを進めることで誰がいつどこを変更しているかだけでなく、タスクの進捗状況を直接会うことなくおおよその内容を把握できる利点に気がついた。
\par  初期段階からGitHub,Gitを使用していればファイルのバージョン管理における混乱を避けられたかもしれない。仮に第4サイクルからGitHub,Gitを使用することになっていたとしても夏季休業中や第3サイクルの時点でGitHub,Gitを何かしらの形で使用しれいればスムーズに第4サイクルでGitHub,Gitを活用して開発を進めることができたと考える。
\bunseki{岩見建汰}