%\chapter{学び}
\section{バージョン管理}
プロジェクト発足時にGitHubを用いたソースコードのバージョン管理についての勉強会に参加した。勉強会ではGitコマンドの使用方法を実践的に学び、第1サイクルではGitHubを使用してバージョン管理を行うことができた。また、Gitで使用するコマンドをマニュアルとして作成しチームで共有した。
\par 第2サイクルは中間発表まで1週間を切っていた、コーディング担当者である2人は日常的に会う頻度が高いメンバー同士だった、GitHubに対する知識が浅く第1サイクル時にマージした際にファイルが消えるなどのトラブルへの対処法が分からないままだったという要因が重なり、Google Drive\footnote{Googleのオンラインストレージ。無料で15GB使用することが可能である。}やMacに搭載されているAirDrop\footnote{AirDropはWi-FiおよびBluetoothを使用して、ファイルの共有を簡単に行えるサービスである。AirDropはApple Inc.の商標である。}を使用してファイル共有を行いコーディングを進めていた。その際は入念にマージや分担について話しあったので大きなトラブルは発生しなかったが、時間を大幅に消費してしまった。その消費してしまった時間をコーディングや成果物の改善に関する議論に費やすことができたら中間成果発表時での成果物の質を高くすることができたと反省している。そして夏季休業終了後の第3サイクルへと突入したが、コーディングを行っていないためGitHubは使用していない。
\par 第4サイクルはGitHubを使用してコーディングを進めることができたが、GitHubを久々に使用するということもあり導入に戸惑ってしまった。原因としてGitHubに関する技術・知識の量がメンバー間で異なっていた事実を早めに解決すべきだったことがあげられる。GitHub,Gitを使用してコーディングを進めることで誰がいつどこを変更しているかだけでなく、タスクの進捗状況を直接会うことなくおおよその内容を把握できる利点に気がついた。
\par  初期段階からGitHubを使用していればファイルのバージョン管理における混乱を避けられたかもしれない。仮に第4サイクルからGitHubを使用することになっていたとしても夏季休業中や第3サイクルの時点でGitHubを何かしらの形で使用していればスムーズに第4サイクルでGitHubを活用して開発を進めることができたと考えられる。
\bunseki{岩見建汰}