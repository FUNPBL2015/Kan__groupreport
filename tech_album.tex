%\chapter{使用技術}
\section{振り返る機能}
振り返る機能は、デフォルトの写真とユーザが撮影した写真の2種類を見ることができるようにViewControllerを2つ用意し、アニメーションなしの画面遷移を行うことで1つの画面で切り替えているように見せている。それぞれのViewControllerにはたくさんの画像などを並べて表示する際によく使用されるUICollectionView及び2つの選択を排他的に行うパーツであるSegmentedを設置し、xibファイルを使用してセルを設定した。以下に振り返る機能で使用した主な技術であるUICollectionView、Segmented,xibについての詳細を記載した。

\begin{description}
\item[UICollectionView]\mbox{} 
\begin{lstlisting}[basicstyle=\ttfamily\footnotesize, frame=single]
1. func collectionView(collectionView: UICollectionView, 
    numberOfItemsInSection section: Int) -> Int {
     if(appDelegate.pic_segmented == 0){  // 撮影写真を選択している場合
       if(DB().getUpdatedCardIDArray().count == 0){
           return 1
       }else{
           return DB().getUpdatedCardIDArray().count  // 枚数分の表示
       }
     }else{  // デフォルト写真を選択している場合
          return DB().cardListSize()
     }
   }
    
2. func collectionView(collectionView: UICollectionView, 
   cellForItemAtIndexPath indexPath: NSIndexPath) -> UICollectionViewCell {
     if(appDelegate.pic_segmented == 0){  撮影写真の場合
      let cell = collectionView.dequeueReusableCellWithReuseIdentifier
      ("PagingCollectionViewCell", forIndexPath: indexPath) as!
      PagingCollectionViewCell
            
     if(DB().getUpdatedCardIDArray().count == 0){
      collectionView.scrollEnabled = false
      let kikoimage = UIImage(named: "kiko.png")
      cell.photo.image = kikoimage
      cell.TitleLabel.text = ""
      cell.introLabel.text = ""
      cell.NoPhotoLabel.text = "撮影写真がありません。\n 写真を撮ると表示されます。"
                
      credit.frame = CGRectMake(70, 310, 300, 120)
      credit.font = UIFont(name: "HiraginoSans-W3", size: 10.0)
      self.addSubview(credit)
      credit.text = " 木古内町観光マスコットキャラクター キーコ"
                
      return cell
     }else{
      var UpdateCardIDArray1 = DB().getUpdatedCardIDArray()
      let NSphotodata1 = DB().getCard(UpdateCardIDArray1
      [indexPath.row]).photo?.photoData
      cell.photo.image = PhotoController().NSSImage(NSphotodata1!)
                
      let titletext1 = DB().getCard(UpdateCardIDArray1
      [indexPath.row]).cardText?.title
      let introtext1 = (DB().getCard(UpdateCardIDArray1
      [indexPath.row]).cardText?.text)!
                
      cell.TitleLabel.text = titletext1
      cell.introLabel.text = introtext1
      cell.NoPhotoLabel.text = ""
                
      return cell
                
      }
     }else{  // デフォルト写真の場合
       self.collectionView.scrollEnabled = true
       let cell = collectionView.dequeueReusableCellWithReuseIdentifier
       ("PagingCollectionViewCell", forIndexPath: indexPath) as! 
       PagingCollectionViewCell
       let cardcounts = DB().cardListSize()
       var IDArray2:[Int] = []
            
       for(var i = 1; i <= cardcounts; i++){
           IDArray2.append(i)
       }
            
       let NSphotodata2 = DB().getDefaultPhoto
       (IDArray2[indexPath.row]).photoData
       
       // 写真データ(NSdata)をimageに変換
       cell.photo.image = PhotoController().NSSImage(NSphotodata2!)
            
       let titletext2 = DB().getCard(IDArray2[indexPath.row]).
       cardText?.title
       let introtext2 = (DB().getCard(IDArray2[indexPath.row]).
       cardText?.text)!
            
       cell.TitleLabel.text = titletext2
       cell.introLabel.text = introtext2
       cell.NoPhotoLabel.text = ""
            
       return cell
      }
  }
 \end{lstlisting}

1. UICollectionViewに表示するセルの数を返すメソッド。撮影写真が0枚の場合は、木古内町観光マスコットキャラクターキーコを表示するために1を返す。0枚以外の場合は、撮影した枚数を返す。 \\
2. UICollectionViewのセルに表示する画像とテキストを設定するメソッド。撮影写真を表示するViewControllerになっている且つ撮影写真が0枚の場合は木古内町観光マスコットキャラクターキーコを設置したセルを返す。撮影写真を表示するViewControllerになっている且つ撮影写真が1枚以上ある場合は、ユーザが撮影した写真を設置したセルを返す。デフォルトの写真を表示するViewControllerになっている場合は、デフォルトの写真を設置したセルを返す。
\end{description}

\begin{description}
\item[Segmented]\mbox{} 
\begin{lstlisting}[basicstyle=\ttfamily\footnotesize, frame=single]
1. @IBAction func segmentedchanged(sender: AnyObject) {
     switch sender.selectedSegmentIndex{
        case 0: // 撮影写真
         appDelegate.pic_segmented = 0
         self.dismissViewControllerAnimated(false, completion: nil)
         
        case 1: // デフォルト写真
         appDelegate.pic_segmented = 1
         
        default:
         break
     }
    }

2. @IBAction func segmentedchanged(sender: AnyObject) {
     switch sender.selectedSegmentIndex{
        case 0: // 撮影写真
         appDelegate.pic_segmented = 0
            
        case 1: // サンプル写真
         appDelegate.pic_segmented = 1
         let targetViewController = self.storyboard!.
         instantiateViewControllerWithIdentifier("album_defaultphoto")
         self.presentViewController( targetViewController, 
         animated: false, completion: nil)
         self.segmented.selectedSegmentIndex = 0         // 選択を戻す
         
        default:
         break
     }
   }
\end{lstlisting}

1. こちらのIBActionはデフォルト写真を表示するViewControllerに設置されているSegmentedに接続されている。撮影写真を選択した場合は該当するViewControllerに画面遷移を行うが、デフォルト写真が選択された場合は何もしない。 \\
2. こちらのIBActionは撮影写真を表示するViewControllerに設置されているSegmentedに接続されている。1のIBActionと逆の動作を行う。 \\

\end{description}

\begin{description}
\item[xib]\mbox{} 
\begin{lstlisting}[basicstyle=\ttfamily\footnotesize, frame=single]
1. class PagingCollectionViewCell: UICollectionViewCell {
    
    @IBOutlet weak var photo: UIImageView!
    @IBOutlet weak var TitleLabel: UILabel!
    @IBOutlet weak var introLabel: UILabel!
    @IBOutlet weak var NoPhotoLabel: UILabel!
    
    override func awakeFromNib() {
        super.awakeFromNib()
        NoPhotoLabel.numberOfLines = 2
        TitleLabel.numberOfLines = 1
        introLabel.numberOfLines = 30
    }
  }
  
2. self.collectionView.registerNib(UINib(nibName: 
   "PagingCollectionViewCell", bundle: nil), forCellWithReuseIdentifier: 
   "PagingCollectionViewCell")
   
   self.addSubview(self.collectionView)
\end{lstlisting}
※xibはstoryboardがアプリ全体の画面遷移を管理するのに対し、1つ又は複数の画面についての設計を行いstoryboardへ組み込むような形で使用する。xibを画面ごとに分割してそれぞれのxibをstoryboardに組み込むような挙動を設定することでチーム開発でのコンフリクトが発生しにくくなり、効率良く開発ができる。 \\ \\
1. xibに使用している部品とソースコード間でのIBOutlet接続及びUILabel内に表示するテキストの行数を設定している。 \\
2. xibを読み込むコードである。xibのファイル名をnibNameにStringで指定しIdentifierも他と重複しない名前をStringで指定する。読み込んだ後は通常のstoryboardと同様にaddSubviewを使用して画面へ追加・表示する。 \\

\end{description}
\bunseki{岩見建汰}