%\chapter{プロジェクトの進め方}
\section{進め方の概要}
ほげほげ
\bunseki{横山翔栄}

\section{開発の進め方}
\subsection{スクラム}
メンバのタスク管理にはスクラムを用いた。以下に今回行ったスクラムの流れについて示す。まず次バージョンまでのプロセスを列挙、細分化し、プロダクトバックログおよびスプリントバックログを作成した。本来は作業の進捗状況を毎日のスクラム会議で確認するところであるが、メンバ間のスケジュール調整の結果、毎週月、水、金曜にスクラム会議を行うこととした。スクラム会議の内容は一般的なスクラム手法と同様に、「前回の会議以降の作業報告」「次回の会議までの作業計画」「作業上の問題点の報告」である。スプリントのタイムボックスは1週間とし、月曜日のスクラム会議を区切りとした。また、本プロジェクトでは、タスクの進捗状況の確認や各メンバの実行中タスクの確認のため、6月中旬からスクラムに加えかんばんも併用することとした。この際、物理的なかんばんを設置する場所が確保できなかったため、WebサービスであるKanbanFlowを利用した。現在のスプリントで行うべきタスクをさらにかんばんによって管理することで、柔軟で正確なタスク分配が可能となり、以後のアプリケーション開発がより計画的に行われるようになった。
\bunseki{横山翔栄}

\subsection{グループのルール}
本プロジェクトでのアプリケーション開発にあたってソースコードのバージョン管理にGitHubを用いたが、これをトラブルなく運用するためメンバ間での取り決めを行った。まずブランチの管理に関してはgit-flowモデル\footnote{Vincent Driessen氏が提唱したGitにおけるブランチ管理規則の一つ}を使用することとし、常に安定版および最新の開発版のソースコードを得られるよう運用した。また、ブランチのマージの際にはプルリクエストに対して3人以上のチェックを行うようにし、バグや動作不良が残ったまま開発が進行することを防止した。
\bunseki{横山翔栄}