%\chapter{到達目標}
\section{本プロジェクトにおける目標}
本プロジェクトは、木古内町へ来訪する観光客の満足度を高めることを目標としている。前述の通り、木古内町の観光情報は分散しており、そのアクセス性は高くない。そこで本プロジェクトでは、観光情報へのアクセス性を高めることによって観光客の負担を減らすほか、観光で得られる体験に付加価値を与えることで、木古内観光の満足度向上を実現する。アプリケーション設計に際しては、木古内町へのフィールドワークで得られた経験を重視し、メンバの実体験に基づくアプリケーション設計を行うことで、より木古内の現状に即したアプリケーションを開発することを目標とした。開発段階ではソフトウェア設計論Iで学習したアジャイル開発手法のうちスクラムを実践することで、敏速なアプリケーション開発を目指した。コーディングにあたっては言語にSwiftを用い、これの習熟に努めるほか、GitやIDE、データベースなどの活用を通じて、アプリケーション開発のための技術を習得することを目標とした。これらに加え、後期ではUI設計やUXの検討など、機能面だけでなくユーザの体験、アプリケーションの使いやすさの面も考慮し、リリースを視野に入れた実用に耐えうるアプリケーションの開発を目標とした。
\bunseki{横山翔栄}