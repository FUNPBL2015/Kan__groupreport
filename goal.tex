%\chapter{到達目標}
\section{本プロジェクトにおける目的}
本プロジェクトは、木古内町へ来訪する観光客の満足度を高めることを目的とするものである。前述の通り、木古内町の観光情報は分散しており、そのアクセス性は高くない。そこで本プロジェクトでは、観光情報へのアクセス性を高めることによって観光客の負担を減らすほか、観光で得られる体験に付加価値を与えることで、木古内観光の満足度向上を実現する。
\bunseki{横山翔栄}

\section{本プロジェクトにおける目標}
前項の目的を達成するため、本プロジェクトでは木古内観光のためのアプリケーションを開発する。アプリケーション設計に際しては、木古内町へのフィールドワークで得られた経験を重視し、メンバの実体験に基づくアプリケーション設計を行うことで、より木古内の現状に即したアプリケーションを開発することを目標とした。開発段階ではソフトウェア設計論Iで学習したアジャイル開発手法のうちスクラムを実践することで、敏速なアプリケーション開発を目指す。コーディングにあたっては言語にSwiftを用い、これの習熟に努めるほか、GitやIDE、WebAPIサービス、各種センサなどの活用を通じて、アプリケーション開発のための技術を幅広く習得していく。
\bunseki{横山翔栄}