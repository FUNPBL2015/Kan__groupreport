%\chapter{背景}
\section{\midorfin{該当分野の現状と従来例}{観光産業と新幹線}}
2015年3月に東京-金沢間を走る北陸新幹線が開業し、北陸の観光産業が活気づいている。それは交通の利便性が向上されたことによって、石川県などの北陸の観光が各種メディアで取り上げられて注目されたことが要因に挙げられる。実際に新幹線開業後のゴールデンウィークには石川県金沢城公園に前年同期比2.5倍もの観光客が訪れたという例もあるほどで、観光産業における新幹線開業の効果は大きい。このように新幹線の開業は、その事業の規模から地域に注目を集め、交通の利便性が向上することによって観光客の増加を促す効果がある。
\bunseki{細川椋太}

\section{木古内町観光の現状}
木古内町は北海道道南地域にある山と海に囲まれた、漁業と酪農及び畜産業を基幹産業とする人口5000人程の自治体である。この木古内町は、2016年春に開業する北海道新幹線の停車駅ができる。それによって観光客の増加が見込まれることから、観光産業にも力を入れている。今現在、木古内町観光の基盤づくりや観光地としての情報発信に特に力を入れており、「観光交流センター・みそぎの郷きこない」の建設や駅前道路の整備を行ったり、木古内町マスコットキャラクターのキーコが木古内駅新幹線観光駅長としてイベントでPRを行うなどの活動をしている。しかし、観光産業に力を入れているという現状はあるが、Webからの観光情報の発信はうまく行われていない。木古内町観光協会のWebサイトで、観光地紹介や特産品紹介などは行われているものの、それが構造上Webサイトの閲覧者にとって見つけづらい状態になってしまっている。そのため、現状木古内町の観光の魅力が木古内町を知りたいと思っている人に伝わらないという問題がある。
\bunseki{細川椋太}

\section{現地調査}\label{sec:gaiyou}
我々は木古内町の観光産業を考えるにあたって、木古内町でフィールドワークを行った。これは、木古内町観光の解決すべき問題点の発見を目的として、我々が観光者の視点に立って行ったものである。気づいた事として、木古内町に着いた際、どこで何ができるのか分かりにくいという問題点を発見した。これは、前述のWebサイトやパンフレットなどに観光情報が分散していて情報が伝わっていないことや、情報提供するコンテンツ自体の問題で木古内の魅力的な部分が十分に伝わっていないことが原因だと考えられる。また、観光を終えた人々には、観光の体験や思い出を友人などの親しい人に伝えたいというモチベーションがあることに気づいた。そして思い出話をする際には、観光中に撮影した写真が観光した場所についての情報を端的に伝えるためのツールとして用いられる。しかし、撮影した写真の整理を行っていない場合、目的の写真を探すことに時間をかけてしまうために、会話が途絶えてしまうといった事例も確認された。
\bunseki{細川椋太}

\section{課題の概要}\label{sec:gaiyou}
このプロジェクトでは、大きく分けて2つの課題の解決に取り組む。まず1つ目は、観光情報が分散しているため、ユーザが欲しいと思った情報へのアクセス性が悪い点である。これによって木古内町の魅力が伝わらず、ユーザが満足するような観光ができていない問題が起きている。2つ目は、観光の体験を話す際に写真を用いながら思い出話をするが、その目的の写真を探すために会話が止まってしまう点である。この原因は、会話が止まってしまうのは写真の整理が行われていないことであると考えられる。その解決のためには、予め写真の整理を行う必要がある。
\bunseki{細川椋太}
