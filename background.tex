%\chapter{背景}
\section{\midorfin{該当分野の現状と従来例}{観光産業と新幹線}}
2015年3月に東京-金沢間を走る北陸新幹線が開業し、北陸の観光産業が活気づいている。それは交通の利便性が向上したことによって、石川県などの北陸の観光が各種メディアで取り上げられて注目されたことが要因に挙げられる。実際に新幹線開業後のゴールデンウィークには石川県金沢城公園に前年同期比2.5倍もの観光客が訪れたという例もあるほどで、観光産業における新幹線開業の効果は大きい。このように新幹線の開業は、その事業の規模から地域に注目を集め、交通の利便性が向上することによって観光客の増加を促す効果がある。
\bunseki{細川椋太}

\section{木古内町と観光}
木古内町は北海道道南地域にある山と海に囲まれた、漁業と酪農及び畜産業を基幹産業とする人口5000人程の自治体である。この木古内町には、2016年春に開業する北海道新幹線の停車駅ができる。それによって観光客の増加が見込まれることから、観光産業にも力を入れている。例えば、木古内町観光の基盤づくりとしては「観光交流センター・みそぎの郷きこない」の建設や駅前道路の整備が行われている。また、観光地としての知名度向上活動に関しては、木古内町マスコットキャラクターのキーコが木古内駅新幹線観光駅長としてイベントでPRを行うなどの活動が行われている。このように、観光産業を盛り上げる活動を積極的に行っている木古内町だが、観光情報は各メディアで分散して発信されており、観光客にとって必要な情報を見つけやすい状態であるとは言いづらい。現在観光情報は、木古内町観光協会のWebサイトやFacebook、パンフレットなど様々なメディアで発信されている。様々なメディアが情報発信を行っているため、観光客は各メディアから多様性のある情報を得ることができる反面、情報の一覧性が低くなってしまっている。また、観光者向けパンフレットはあるものの、我々が2015年5月に1度目の木古内町フィールドワークを行った際には、木古内駅構内など観光客が簡単に見つけることのできる場所に置かれていなかった。そのため、その場で観光情報を得ることが難しくなってしまっていた。これらの要因から、木古内町の観光の魅力が観光客や木古内町を知りたい人に伝わりづらいという問題がある。
\bunseki{細川椋太}

\section{課題の概要}\label{sec:gaiyou}
我々はアプリ開発を行うにあたり、大きく分けて2つの課題の解決に取り組む。まず1つ目は、観光情報のアクセス性改善と魅力の発信である。先述の、観光情報が各種メディアに分散しているため観光情報の一覧性が低くなってしまっている点を解決し、その情報を効果的に伝えることによって木古内町観光を魅力的に伝える。2つ目の解決する課題は、観光した際に撮影した写真を振り返る機会が少ないことである。観光中にスマートフォンで撮影した写真を整理をしたり振り返るという機会があまりないという観光客は少なくないだろう。しかしこの課題を解決し写真を振り返る機会を増やすことができた場合、木古内町を思い返し再度観光へ行くリピーターの増加や、写真を通した思い出話から広がる口コミによって木古内町の認知度が高まることに繋がると考えられる。本プロジェクトでは上記2点の課題へ、観光アプリを開発することによってアプローチし、木古内町観光の満足度向上を図る。
\bunseki{細川椋太}
