%\chapter{開発準備}
\section{開発に使用したツール}

iOSアプリケーションを開発する上で使用する言語はSwiftを選択した。Swiftとは、iOSとMac向けのアプリケーションを開発するためにAppleが提供しているプログラミング言語である。開発したソースコードのバージョン管理ツールとして、Git/GitHubを使用した。Gitはソースコードやファイルの変更履歴を管理することができるシステムである。Gitは変更履歴をリポジトリと呼ばれる場所に保存する。リポジトリには各メンバの開発環境に作成するローカルリポジトリと、メンバで変更履歴を共有するリモートリポジトリの2種類が存在するが、GitHubはこのうちリモートリポジトリを提供するサービスの一つである。各メンバの作業内容、すなわちソースコードの変更履歴をリモートリポジトリで共有、統合することができるため、複数のメンバが同時に開発を進めることができる。タスク管理には、KanbanFlowというWebサービスを使用した。KanbanFlowはオンライン上で利用できる無料のタスク管理システムである。KanbanFlowはやらなくてはならないタスクを作業予定、作業中、作業完了の3つの段階に分ける。また、そのタスクそれぞれに担当者を記入する。これにより、メンバーのタスクが今どのような状態なのかをチームで共有することが可能になった。
\bunseki{池田俊輝}

\section{勉強会}
SwiftとGit/GitHubの勉強会を5月に週に1回、1ヶ月かけて行った。この勉強会ではTAの方々から指導してもらった。進め方として1時間ほどTAがスライドを用いて講義を行い、その後2時間程度演習形式でTAに用意してもらった例題を我々が解いた。勉強会中にわからないところがあれば、TAに個別に対応してもらった。アプリ開発の勉強会ではXcodeというiPhoneアプリケーション開発用デベロッパーツールの基本的な使い方を学んだ。そこでストーリーボードの使い方や、デバッグするためにMac上でiPhoneやiPadの環境をシミュレートする方法等を学んだ。具体的にはアプリ内でマップを表示し、自分の指定した場所にピン型のマーカーを設置し、その場所をタップすると自分の指定したWebサイトに遷移するアプリを作る演習を行った。
Git/GitHubの勉強会ではバージョン管理システムの目的や、管理方法の基礎を学んだ。SwiftとGit/GitHubを使用した開発の練習を目的として、サンプルアプリの開発を行った。実際にプログラムの構造を考えてどの機能の実装を誰が行うのか決めて、ソースコードのファイルを共有しながら各自がコーディングを行うといった全体の流れを確認した。

\bunseki{池田俊輝}