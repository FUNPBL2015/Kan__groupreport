%\chapter{開発準備}
\section{使用言語とプラットフォーム}

iOSアプリケーションを開発する上で使用する言語はSwiftを選択した。Swiftとは、iOSとMac向けのアプリケーションを開発するためにAppleが作ったプログラム言語である。アプリを開発する言語としてObjective-CとSwiftのふたつがあるが、
コーディングが苦手な人でも直感的に開発をすることができるためObjective-Cではなくこちらを採用した。開発したソースコードのバージョン管理ツールとして、Git/GitHubを使用した。Git/GitHubは、プログラムのソースコードなどの変更履歴を記録・追跡するための分散型バージョン管理システムである。これにより、チームで共有して作業しているファイルにどのような変更をしたのか確認できる。同時に複数ユーザが編集してしまうために、先に編集した人の変更内容が消える問題などを防ぐことができた。タスク管理には、KanbanFlowを使用した。タスクを作業予定、作業中、作業完了の3つの段階に分けることで、進捗具合を一目で確認することができる。これにより、メンバーのタスクが今どのような状態なのかチームで共有することが可能になった。
\bunseki{池田俊輝}

\section{勉強会}
SwiftとGit/GitHubの勉強会を5月に週に1回、1ヶ月かけて行った。講師はTAの方々にお願いした。Swiftのアプリ開発では最初にXcodeの基本的な使い方を学んだ。そこでストーリーボードの使い方、Mac上でのシミュレートの方法等を演習形式で学んだ。具体的にはアプリ内でマップを表示し、自分の指定した場所にピン型のマーカーを設置し、その場所をタップすると自分の指定したWebサイトに遷移するアプリを作る演習を行った。
Git/GitHubにおいては、バージョン管理システムの目的や、管理方法の基礎を学んだ。SwiftとGit/GitHubを使用した開発の練習を目的として、サンプルアプリの開発を行った。実際にプログラムの構造を考えてどの機能の実装を誰が行うのか決めて、プログラムファイルを共有しながら各自がコーディングを行うといった全体の流れを確認した。

\bunseki{池田俊輝}