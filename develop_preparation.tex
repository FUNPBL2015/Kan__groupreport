%\chapter{開発準備}
\section{開発に使用したツール}

iOSアプリケーションを開発する上で使用する言語はSwiftを選択した。Swiftとは、iOSとMac向けのアプリケーションを開発するためにAppleが用意したプログラム言語である。アプリを開発する言語としてObjective-CとSwiftがあるが、
Swiftは簡単に実装することができ、直感的に開発をすることができるためObjective-Cではなくこちらを採用した。開発したソースコードのバージョン管理ツールとして、Git/GitHubを使用した。Gitはソースファイルの変更履歴を管理することができるシステムである。Gitはリポジトリ(管理保存場所)先として、ローカル、もしくはリモートを選択出来るが、GitHubはそのリモート先にあたるものであり、プロジェクトの管理をWeb上で行うことが出来るサービスである。これにより、チームで共有して作業しているファイルにどのような変更をしたのか確認できる。タスク管理には、KanbanFlowというシステムを使用した。KanbanFlowはネット上で利用できる無料のタスク管理システムである。KanbanFlowはやらなくてはならないタスクを作業予定、作業中、作業完了の3つの段階に分ける。また、そのタスクそれぞれに担当者を記入する。これにより、メンバーのタスクが今どのような状態なのかをチームで共有することが可能になった。
\bunseki{池田俊輝}

\section{勉強会}
SwiftとGit/GitHubの勉強会を5月に週に1回、1ヶ月かけて行った。講師はTAの方々から指導してもらった。進め方としてTAがスライドを用いて講義を行い、その後に演習形式でTAが例題を用意してもらい我々がその問題を解いた。勉強会中にわからないところがあれば、TAが個別に対応してもらった。Swiftのアプリ開発ではXcodeというiPhoneアプリケーション開発用デベロッパーツールの基本的な使い方を学んだ。そこでストーリーボードの使い方やデバックするためにMac上で、iPhoneやiPadの環境をシミュレートする方法等を演習形式で学んだ。具体的にはアプリ内でマップを表示し、自分の指定した場所にピン型のマーカーを設置し、その場所をタップすると自分の指定したWebサイトに遷移するアプリを作る演習を行った。
Git/GitHubの勉強会ではバージョン管理システムの目的や、管理方法の基礎を学んだ。SwiftとGit/GitHubを使用した開発の練習を目的として、サンプルアプリの開発を行った。実際にプログラムの構造を考えてどの機能の実装を誰が行うのか決めて、プログラムファイルを共有しながら各自がコーディングを行うといった全体の流れを確認した。

\bunseki{池田俊輝}