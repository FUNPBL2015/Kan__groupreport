%\chapter{使用技術}
\section{カメラ}
カメラ機能の実装の方法として``AVFoundation Framework"を使うか``UIImagePickerController"を使うか2パターンあった。最初に、``AVFoundation Framework"というSwiftのFrameworkを使用して実装しようとした。このFrameworkは開発者がカメラの画素数やピントの調整のやり方などを全てマニュアルで制御しなくてはならなかった。とくにピントの調整においては、撮影の対象物とiPhoneのカメラの距離から自分で計算式を考えてピントが合うようにしなければならずうまく実装することができなかった。次に``AVFoundation Framework"を使わず``UIImagePickerController"というクラスを用いて実装した。このクラスを用いるとピントの調整や明るさの調整はアプリ側がオートで処理をしてくれるため簡単にカメラで写真をとる機能を実装することができた。そして、撮影した写真はUIImageとして保存されデータベースに書き込むことができる型ではなかったため、UIImageデータをNSDataに変換してアプリ内のデータベースに保存することにした。これにより、他のクラスで撮った画像を使用する場合もアクセスが可能になった。以下にそれぞれかくとなるコードをピックアップして記述した。

\begin{description}
\item[PhotoController]\mbox{} 
\begin{lstlisting}[basicstyle=\ttfamily\footnotesize, frame=single]
1. func ImageNSS(image:UIImage) -> NSData? {
        let data:NSData = UIImageJPEGRepresentation(image,0.8)!
        return data
    }
    
2. func NSSImage(data:NSData) -> UIImage?{
        let decodeSuccess = data
        let img = UIImage(data: decodeSuccess)
        return img
    }

3. func camerastart(){
        let sourceType:UIImagePickerControllerSourceType 
        	  = UIImagePickerControllerSourceType.Camera
        if UIImagePickerController.isSourceTypeAvailable
          (UIImagePickerControllerSourceType.Camera){
            let cameraPicker = UIImagePickerController()
            cameraPicker.sourceType = sourceType
            cameraPicker.delegate = self
            self.presentViewController(cameraPicker, 
            animated: true, completion: nil)
        }
    }

4. func imagePickerController(imagePicker: UIImagePickerController, 
        didFinishPickingMediaWithInfo info: [String : AnyObject]) 
        if let pickedImage = info[UIImagePickerControllerOriginalImage] 
        as? UIImage {
            let image:UIImage! = pickedImage
            let appDelegate:AppDelegate = UIApplication.
            sharedApplication().delegate
             as! AppDelegate
            let pic_id = appDelegate.P_ID
            if(image != nil){
                DB().addPhoto(pic_id!, photoData: ImageNSS(image!)!)
                let photoID = DB().getLastPhotoID()
                DB().linkToCard(photoID)
            }
        }
        imagePicker.dismissViewControllerAnimated(true, completion: nil)
    }
 \end{lstlisting}

1.UIImage型の撮った画像をNSData型に変換するメソッド。データベースで扱える型の中にNSData型があったため保存する型として選択した。引数としてUIImage型の撮った画像データを受け取り、80%圧縮をかけてNSData型に変換して返す。\\
2.NSData型のデータをUIImage型のデータに変換するメソッド。データベースに保存された画像データはNSData型なので画像として表示するためにはUIImage型に変換する必要がある。\\
3. UIImagePickerControllerを呼び出すメソッド。まず、iPhoneのカメラが利用可能か調べる。もし、利用可能なら現在の画面からカメラの撮影画面に遷移する。\\
4.imagePickerControllerはAppleが用意しているメソッドで、写真撮影ボタンを押して撮影が完了するときに呼び出される。まず、カードリストで選択したIDを取得する。データベース上から、その取得したIDに対応した画像データ項目に撮った画像データを書き込む。以上の処理が終わるとimagePickerControllerを閉じる。\\


\end{description}

\bunseki{池田俊輝}