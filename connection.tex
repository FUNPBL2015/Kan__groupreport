\section{ステークホルダー会議について}
我々はキーコ紀行をリリースするにあたって木古内町長と木古内商工会 木古内町観光協会へ最終成果物について報告すると共にキーコ紀行の運用について会議を行う予定である。また、前期にレビューを頂いた「道南いさりび鉄道(株)」の方にも同様に報告を行い、レビューを頂きたいと考えている。以下に現段階で考案している報告内容と会議内容を示す。
\begin{description}
 \item[報告内容]\mbox{}
 	\begin{itemize}
	\item 作成背景
	\item 観光客と木古内町のメリット・デメリット
	\item キーコ紀行の使い方を説明(プリンター持参)
	\item キーコ紀行のユーザストーリ \\
	\end{itemize}
 \item[会議内容]\mbox{}
	\begin{itemize}
	\item キーコ紀行内に記載している店舗情報をお店から直接収集することについて
	\item キーコ紀行で使用している写真の使用許可について
	\item 「木古内観光交流センターみそぎの郷きこない」にプリンターを置いての運用について
	\item キーコ紀行内の観光情報をアップデートする運用について
	\item 観光スポット紹介文の是非について
	\item その他、キーコ紀行全般の是非について
	\end{itemize}
\end{description}

\bunseki{岩見建汰}