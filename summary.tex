%\chapter{まとめ}
\section{まとめ}
本プロジェクトでは、フィールドを調査することによって問題点を発見し、その問題点を解決するためのアプリケーションを開発して、地域・社会に貢献することを目標として活動を行った。開発には、その目標を達成する質の高いアプリを開発するために評価と改善を繰り返すアジャイル開発手法を用いた。
\par 観光系グループでは、木古内町の観光をフィールドと設定して調査を行い、その調査結果をもとに分析を行って「観光情報のアクセス性改善と魅力の発信」「観光した際に撮影した写真を振り返る機会が少ない」といった課題を定義して、それを解決するためにiOSアプリ「キーコ紀行」を開発した。このアプリは、「木古内町でできること」に着目した観光情報の利用と、観光スポットで撮影した写真を用いてオリジナルリーフレットの作成ができるアプリケーションである。このアプリを開発する過程で4つのPDCAサイクルを経て、その活動経験とたくさんの評価やアドバイスから、グループメンバそれぞれが要件定義や実装、マネジメントなどプロジェクト活動全体に関する数多くの学びを得ることができた。
\par 今後、この「キーコ紀行」は観光グループのメンバが引き続き開発を行い、2016年2月中にリリースする予定である。このアプリをリリースをするために未実装の機能の実装を行い、その後に木古内町長と木古内商工会「木古内町観光協会」へ最終成果物について報告するとともにキーコ紀行の運用について相談を行う予定である。リリース後には実際に木古内でこの「キーコ紀行」を使っていただいた方からのレビューなどを受け、よりよいアプリへと改善するとともに継続的な運用も行う予定である。
\bunseki{細川椋太}
