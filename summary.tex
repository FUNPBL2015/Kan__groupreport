%\chapter{まとめ}
\section{まとめ}
本プロジェクトでは、フィールドを実際に調査してそこから問題点を見つけ、そこで見つかった問題点を解決するためにiOSアプリケーションを開発して、それにより地域・社会に貢献することを目標として活動を行った。開発には、その目標を達成する質の高いアプリを開発するために評価と改善を繰り返すアジャイル開発手法を用いた。\\
観光系グループでは、木古内町の観光をフィールドと設定して調査を行い、その調査結果をもとに分析を行って「観光情報のアクセス性改善と魅力の発信」「観光した際に撮影した写真を振り返る機会が少ない」といった課題を定義して、それを解決するためにiOSアプリ「キーコ紀行」を開発した。このアプリは、「木古内町でできること」に着目した観光情報の利用と、観光スポットで撮影した写真を用いてオリジナルのリーフレットを作成できるアプリケーションである。このアプリが今に至るまでに大きく4つのPDCAサイクルを経て、その過程とたくさんの評価やアドバイスを頂いた経験から、グループメンバそれぞれが要件定義や実装、マネジメントなどプロジェクト活動全体に関する数多くの学びを得ることができた。\\
今後、この「キーコ紀行」は観光グループのメンバが引き続き開発を行い、2016年3月の北海道新幹線開業までにはリリースする予定である。リリースをするために、木古内町の方からのレビューを受けて内容を改善することや、未実装部分の機能の実装を行う。リリース後には実際に木古内でこの「キーコ紀行」を使っていただいた方からのレビューなどを受け、よりよりアプリへと改善するとともに継続的な運用も行う予定である。
\bunseki{細川椋太}
